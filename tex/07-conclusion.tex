\chapter{Conclusion} \label{chap:conclusion}

The dGMU model offers a promising framework for learning fusion representations from multi-platform genomic data. The main objective of this work is to utilize dimensionality reduction and interpretable features for multimodal cancer phenotype prediction. In experiment, the dGMU model was able to learn a biologically relevant latent space using RNA expression and copy number variation from eight cancer types, and it outperformed both unimodal features and various common fusion strategies in classification agreement. The results indicate that deep learning architectures based on GMUs have the potential to expedite representation learning and knowledge integration in the study of cancer pathogenesis. Additional evaluations must be made to test further if the dGMU latent space can produce features that can generalize associations between increasingly diverse biomedical data. This effort could include the identification of potential cross cancer biomarkers by integrating heterogeneous cancer data in ensembles. In future work, we expect these kinds of models to support integrative and interpretable deep learning methods. As integrative machine learning methods become more common, we believe that multiplicative gating systems will provide clinicians with viable models to personalize patient care to their unique genomic profile.


The LIME algorithm was extended to facilitate the interpretation of multi-platform genomic data. We demonstrated the use of this algorithm on a multimodal neural network to generate gene-wise RNA-seq and SNV explanations for the classification of correctly labelled instances. We found that gene-wise explanations are useful for revelling clinically relevant genes used by the machine learning model to make accurate predictions. We also demonstrated that the explanations derived from multi-platform genomic data is helpful for identifying potential biomarkers and validating the predictive influence of known oncogenes. The additonal insight gained by examing the explanations is helpful to gain trust in the predictions of the dGMU model. For a given classification, a domain expert can obtain the relative contributions of the modalities and the top explanatory RNA-seq expression and SNV gene regions. In the future, we would like to evaluate enhanced interpretable representations that incorporates the interaction between modalities. This involves incorporating known pathways and gene-gene relationships as a part of the model. We believe that correlating deeper biological relationships will help facilitate a greater insight of the underlying machine learning model.