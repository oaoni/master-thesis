\chapter{Introduction}\label{chap:introduction}

Recent advances in biotechnology have enabled a multidimensional approach for exploring human disease. New high-throughput technologies can quantify and characterize the biomolecules that define the architecture, behaviour, and dynamics of a biological system. Research in the last decade has introduced a multifaceted exploration of cancer biology at an unprecedented scale \cite{kashyap2015big}. Cancer research projects are using next-generation sequencing (NGS) data to characterize the genome, epigenome, and transcriptome to capture the complexity and phenotypic heterogeneity of cancer cells \cite{encode2012integrated}. 

Despite the availability of wide-ranging forms of genomic data, few predictive models are designed to leverage multidimensional data sources. The unilateral approach has been effective in the identification of single-cell expression profiles, epigenomic states, and previously uncharacterized sequence variants \cite{kundaje2015integrative,huang2017fast}. However, individual data types are unable to capture the systems-wide view required to understand the complexity of cancer pathology. This has resulted in the need for integrative methods designed to combine multi-platform data \cite{zitnik2019machine}. In the most recent biomedical literature there has been several attempts to utilize multi-platform biomedical data, however these existing methods rely on linearly fusing data sources, resulting in high data sparsity and an inability to capture both intra-modality and cross-modality correlations \cite{sun2018multimodal,liang2015integrative,le2010integrative}. In this work, we introduce a deep learning approach based on the Gated Multimodal Unit (GMU) to facilitate the integration of multi-platform genomic data and predict cancer cell tissue sub-class. GMUs are neural networks that utilize multiplicative gates to learn intermediate representations between diverse sources of information. Here we show that a series of deeply connected GMUs can be used to extract a biologically relevant latent space from multi-platform genomic data. 

The presence of diverse forms of genomic data has encouraged the use of machine learning methods in clinical decision support. Clinicians are increasingly adopting coupled frameworks of NGS and predictive models to support cancer diagnosis and patient stratification. Rapid developments in machine learning is enabling opportunities for improved clinical decision making in the healthcare industry, however, several key challenges hinder its utility by clinicians and researchers. The application of deep learning for medical predictions often results in a hindered ability to interpret the decision made by the classifier. Healthcare professionals require informative tools that can explain their predictions. Domain experts need to ensure a level of trust in predictive models by evaluating the usefulness, reliability, and internal logic of the system. In this work, we assess several methods that leverages interpretable dimensionality reduction and model agnostic explanations to help understand the behaviour of the underlying model.

The remaining content of this thesis is organized as follows. \hyperref[chap:background]{Chapter 2} provides a general background of machine learning, and in particular reviews the basic concepts of deep learning, autoencoders, and model generalization. \hyperref[chap:relatedwork]{Chapter 3} describes the most recent academic literature pertaining to neural network based dimensionality reduction and information fusion, and model interpretation in biomedical research. \hyperref[chap:deepgmu]{Chapter 4} describes the main components and architecture of our approach, the dGMU (deep gated mulitmodal unit) for integrative information fusion and representation learning. In \hyperref[chap:materials]{chapter 5}, we provide a description of the methodologies utilized to evaluate our approach, and in \hyperref[chap:results]{chapter 6} we report the results obtained across a variety of experiments. Lastly, the thesis is concluded in \hyperref[chap:conclusion]{chapter 7} with a summary of the experimental results and a discussion of potential future research directions.


