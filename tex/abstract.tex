\begin{abstract}
	The abundance of next-generation sequencing (NGS) data has encouraged the adoption of machine learning methods to aid in the diagnosis and treatment of human disease. In particular, the last decade has shown the extensive use of predictive analytics in cancer research due to the prevalence of rich cellular descriptions of genetic and transcriptomic profiles of cancer cells. Despite the availability of wide-ranging forms of genomic data, few predictive models are designed to leverage multidimensional data sources. Many existing methods rely on linearly fusing data sources, resulting in high data sparsity and an inability to capture both intra-modality and cross-modality correlations. In this paper, we introduce a deep learning approach based on the Gated Multimodal Unit (GMU) to facilitate the integration of multi-platform genomic data, and the prediction of cancer cell sub-class and patient prognosis. GMUs are neural networks that utilize multiplicative gates to learn intermediate representations between diverse sources of information. Here we show that a series of deeply connected GMUs can be used to extract a biologically relevant latent space from multi-platform genomic data. Experimental results on nine cancer types and four forms of NGS data (copy number variation, simple nucleotide variation, RNA expression, and miRNA expression) showed that GMU based models improved the classification agreement of unimodal approaches and outperformed other fusion strategies in class accuracy and survival regression. The results indicate that deep learning architectures based on GMUs have the potential to expedite representation learning and knowledge integration in the study of cancer pathogenesis.
\end{abstract}